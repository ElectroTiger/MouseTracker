\documentclass{article}
\usepackage{geometry}
\geometry{letterpaper, portrait,margin=1in}
\usepackage[T1] {fontenc}
\usepackage[scaled=1.1] {garamondx}
\usepackage[garamondx,cmbraces] {newtxmath}
\usepackage{setspace}
\usepackage[os=win]{menukeys}
\usepackage[tikz]{bclogo}
\onehalfspacing

\usepackage{listings}
\usepackage{color}

\definecolor{dkgreen}{rgb}{0,0.6,0}
\definecolor{gray}{rgb}{0.5,0.5,0.5}
\definecolor{mauve}{rgb}{0.58,0,0.82}

\lstset{frame=tb,
	language=C++,
	aboveskip=3mm,
	belowskip=3mm,
	showstringspaces=false,
	columns=flexible,
	basicstyle={\small\ttfamily},
	numbers=none,
	numberstyle=\tiny\color{gray},
	keywordstyle=\color{blue},
	commentstyle=\color{dkgreen},
	stringstyle=\color{mauve},
	breaklines=true,
	breakatwhitespace=true,
	tabsize=3
}

\usepackage{hyperref}
\usepackage{vhistory}
\begin{document}	
	\title{Setting Up ssh to Raspberry Pi Model B}
	\author{Weimen Li}
	\date{November 19th, 2016}
	\maketitle
	\tableofcontents
	\section{Introduction}
	In order to ease development of the system, it is useful to allow the Raspberry Pi 3 Model B+ to be connected to your personal computer without the need for an external monitor, mouse, or keyboard. This allows you to develop on the Raspberry Pi anyplace where you can get access to WiFi, with minimal need for cables. 
	
	This guide explains how to set up the Raspberry Pi Model B+ for this sort of development. It uses ssh to connect to the Raspberry Pi.
	
	\section {Connecting to the Raspberry Pi}
	
	If you are on Windows, install MobaXTerm. This software allows you to both use ssh to connect to various systems as well as employ the Linux command prompt for various commands. Using the Linux  prompt allows the following instructions to apply equally well whether you are running Windows, Linux, or Mac.
	
	Having installed MobaXTerm, using ssh simply involves opening a local terminal, then typing \verb|ssh user@ip/hostname|, a typical example being \verb|pi@192.168.1.200| or \verb|pi@myraspberry.local|
	
	\subsection{For WiFi Connection}
	 Using a WiFi connection to connect the Raspberry Pi to your personal network is the easiest way to set up a headless connection. You simply need to create a textfile on the SD card according to the instructions below, which were adapted from \href{https://www.raspberrypi.org/blog/another-update-raspbian/}{Raspberry Pi Blog: The Latest Update to Raspbian}
	 
	 \begin{enumerate}
	 	\item Insert the SD card with the Raspbian OS into your computer.
	 	\item Create a new file named \verb|wpa_supplicant.conf|. The file should have the extension \verb|.conf|; on some computers, you may need to enable an option to show file extensions before you can change it.
	 	\item Follow the instructions here: \href{http://weworkweplay.com/play/automatically-connect-a-raspberry-pi-to-a-wifi-network/}{Automatically connect a Raspberry Pi to a Wifi network} concerning the \verb|wpa_supplicant.conf| file to add the correct lines to the file you just created.
	 	\item Also follow the instructions in the link above to assign a static IP address to your Raspberry Pi, which you will later use to ssh into it.
	 	\item Replace the SD card into your Raspberry Pi, insert the WiFi dongle, and attach the power supply. It should automatically connect to your wireless network.
	 \end{enumerate}
	
	\subsection{For Ethernet Connection}
	In case connecting to WiFi is not so simple, such as when your WiFi network requires a webpage-based login or registering a device in a database, one option is to perform a direct ethernet connection between your laptop and the Raspberry Pi. This approach also allows you to share the internet connection between your computer and the Raspberry Pi.
	
	\subsubsection{Option 1: Change the hostname}
	\begin{enumerate}
		\item Install Bonjour Print Services from Apple.
		\subitem Windows Link: \url{https://support.apple.com/kb/DL999?viewlocale=en_US&locale=en_US}
		\item Follow the instruction at \href{http://www.dexterindustries.com/howto/change-the-hostname-of-your-pi/}{Dexter Industries: Change the Hostname of your Pi} to change the hostname on the SD card to something of your choosing.
		\item Replace the SD card into your Pi, connect it to your computer using the ethernet cable, and attach the power source.
		\item ssh into the Raspberry Pi using \verb|ssh pi@myhostname.local| - you should not have access to the terminal!

	\end{enumerate}
	
	\subsubsection{Option 2: Change the IP Address}
	Changing the hostname is the easiest and most flexible way to access the Raspberry Pi through a direct ethernet connection. In case that doesn't work, the below instructions will also connect you to the Raspberry Pi.
	The instructions are adapted from this source: \href{https://www.raspberrypi.org/forums/viewtopic.php?f=91&t=24993}{Raspberry Pi forums: SSH over direct ethernet connection}
		\begin{enumerate}
			\item Insert the Raspberry Pi's microSD card into your computer using an adapter, as needed. Open the microSD card and append the following line into cmdline.txt:
			\begin{lstlisting}
			ip=192.168.1.200::192.168.1.1:255.255.255.0:rpi:eth0:off
			\end{lstlisting}
			So that the total text looks something like: 
			\begin{lstlisting}
			dwc_otg.lpm_enable=0 console=ttyAMA0,115200 kgdboc=ttyAMA0,115200 console=tty1 root=/dev/mmcblk0p2 rootfstype=ext4 elevator=deadline rootwait ip=192.168.1.200::192.168.1.1:255.255.255.0:rpi:eth0:off
			\end{lstlisting}
			You MUST ensure that there is no new line after the new entry. In other words, do not hit the "Enter" key on your keyboard after pasting the line in.
			\item Now change the IP address settings for your ethernet connection:
			\subitem Windows: Navigate to \verb+"Control Panel\Network and Internet\Network Connections"+ and right clicking on the ethernet connection that you have connected your Raspberry Pi to, go to \menu{Properties>Internet Protocol Version 4>Properties>Use the following IP Address}
			\item Use the following IP address settings:
			\subitem IP Address: 192.168.1.100
			\subitem Subnet Mask: 255.255.255.0
			\subitem Default gateway: 192.168.1.1
			\item Open your terminal (or MobaXTerm) and ssh to pi@192.168.1.200 with the default password, which is "raspberry". 
			\item You have now successfully ssh'd into the Raspberry Pi computer! 
% The following commented lines were removed because it doesn't work, though the original forum post suggests that you do it.
%			\item Type \verb|cd /etc/network/| and hit \menu{\keys{\enter}}.
%			\item Type \verb|sudo nano interfaces| and hit \keys{\enter}.
%			\item Using the arrow keys, navigate to the line that says \verb|iface eth0 inet custom| and replace it with:
%			\begin{verbatim}
%				iface eth0 inet static
%				address 192.168.1.200
%				netmask 255.255.255.0
%				network 192.168.1.0
%				broadcast 192.168.1.255
%				gateway 192.168.1.1
%			\end{verbatim}
%			\item Press \keys{\ctrl + O} to open the save menu and hit \keys{\enter} to save.
%			\item Press \keys{\ctrl + X} to return to the terminal.
%			\item Enter \verb|cd ~| to return to your home folder.
%			\item Type \verb|sudo shutdown -h now| and wait until the lights stop blinking to shut down the Raspberry Pi.
%			\item Unplug the Raspberry Pi, insert the SD card back into your computer, and remove the \verb|ip=192.168.1.200::192.168.1.1:255.255.255.0:rpi:eth0:off| line which you added to cmdline.txt.
%			\item Replace the SD card into the Raspberry Pi and ssh into it again. If it still works, then you have done everything correctly.
		\end{enumerate}
		
		\subsection{Sharing your Internet Connection with the Raspberry Pi}
		Supposing that it is difficult to connect your Raspberry Pi to WiFi, but you have internet access on your personal computer, it is possible to share your internet connection with your Pi. The below instructions pertain to Windows only, though it should also be possible on other operating systems.
		
		\begin{enumerate}
			\item Install Bonjour Print Services from Apple.
			\subitem Windows Link: \url{https://support.apple.com/kb/DL999?viewlocale=en_US&locale=en_US}
			\item Assign a hostname to your Raspberry Pi if you haven't done so already, using the instructions at \href{http://www.dexterindustries.com/howto/change-the-hostname-of-your-pi/}{Dexter Industries: Change the Hostname of your Pi}. 
			\item Navigate to \verb+"Control Panel\Network and Internet\Network Connections"+ and right click connection that has internet access.
			\item Go to \menu{Properties>Sharing}, check both boxes, and hit "OK". 
			\item Restart your Raspberry Pi. You may also need to restart your personal computer.
			\item ssh into your Pi and type \verb|ping google.com| to verify internet access. You should get output like:
			\begin{verbatim}
				PING google.com (172.217.2.14) 56(84) bytes of data.
				64 bytes from lga15s45-in-f14.1e100.net (172.217.2.14): icmp_seq=1 ttl=56 time=6.30 ms
				64 bytes from lga15s45-in-f14.1e100.net (172.217.2.14): icmp_seq=2 ttl=56 time=7.07 ms
				64 bytes from lga15s45-in-f14.1e100.net (172.217.2.14): icmp_seq=3 ttl=56 time=7.09 ms
			\end{verbatim}
			This indicates you have internet access. The ping program will continue by itself for a while after you verified internet access exists, so press \keys{\ctrl+C} to exit it.
		\end{enumerate}
		
		\section{First Time Setup}
		This section details what you should do with your Raspberry Pi if it is your first time connecting to it. This assumes that you have access to the raspberry pi terminal already, either by attaching a monitor, mouse, and keyboard, or using ssh to connect to it "headlessly". You should also have an internet connection on your Raspberry Pi, either through connecting it to your network or sharing the internet connection from your PC.
		\begin{enumerate}
			\item If the "Raspi-Config" menu hasn't appeared automatically, type "sudo raspi-config" into the terminal.
			\item Select \menu{1 Expand File System}. A confirmation should appear. Confirm it and return to the configuration menu.
			\item Select the option to \menu{2 Change User Password} and change the password to something secure. 
			\begin{bclogo}{Warning}
				Do not forget your password! Write it down if you need to.
			\end{bclogo}
			\item Select the option to \menu{6 Enable Camera} and enable it.
			\item Select the option to \menu{Finish} at the bottom, and confirm that you wish to reboot the Raspberry Pi. The Rpi will then reboot.
			\item Run:
			\begin{verbatim}
				sudo apt-get update
				sudo apt-get upgrade
			\end{verbatim}
			Some prompts may appear during the upgrade command. Confirm them as they appear.
		\end{enumerate}
		\subsection{Installing Remote Desktop}
		If you are using running the Raspberry Pi headless, you may wish to use the desktop GUI for the Raspberry Pi. This details how to do that.
		
		The following instructions are adapted from \href{https://www.raspberrypi.org/documentation/remote-access/vnc/README.md}{Raspberry Pi: VNC}.
		
		\begin{itemize}
			\item On your Pi, run \verb|sudo raspi-config| and navigate to \menu{Advanced Options > VNC > Yes} to activate the VNC server.
			\item Type \verb|vncserver| to start the VNC server on your Pi. 
			\item On your PC, download VNC Viewer at \url{https://www.realvnc.com/download/viewer/}.
			\item Install the program and run it. Connect to the Raspberry Pi either through the hostname you set or the IP address that appeared when you ran the \verb|vncserver| command on your Raspberry Pi.
			\item Optional: Increase the resolution of the connection (Adapted from \url{https://support.realvnc.com/knowledgebase/article/View/523})
			\subitem \verb|sudo nano /boot/config.txt|
			\subitem Add the following lines to the file:
			\begin{verbatim}
				# Custom settings to force a resolution - Weimen
				hdmi_force_hotlpug=1
				hdmi_ignore_edid=0xa5000080
				hdmi_group=2
				hdmi_mode=73
			\end{verbatim}
			Where \verb|html_mode| is set according to the resolution you want according to \url{https://www.raspberrypi.org/documentation/configuration/config-txt.md} under the \verb|These values are valid if hdmi_group=2 (DMT):| line.
			\subitem Hit \keys{\ctrl+O} to save, and \keys{ctrl+X} to exit.
			\subitem Reboot your pi with \verb|sudo reboot|.
			\subitem Restart the VNC server with \verb|vncserver| and connect to it using the VNCViewer program on your PC. The resolution should have increased to what you set it to.
			
			
		\end{itemize}
		
		\section{Optional: Install Development Toolchains}
		\subsection{Install gcc 6.2}
		\url{https://solarianprogrammer.com/2016/06/29/raspberry-pi-raspbian-compiling-gcc-6/}
		\begin{verbatim}
			sudo apt-get update
			sudo apt-get upgrade
			cd ~
			wget mirrors-usa.go-parts.com/gcc/releases/gcc-6.2.0/gcc-6.2.0.tar.bz2
			tar xf gcc-6.2.0.tar.bz2
			cd gcc-6.2.0
			contrib/download_prerequisites
			cd ~
			mkdir build && cd build
			../gcc-6.2.0/configure -v --enable-languages=c,c++ --prefix=/usr/local/gcc-6.2.0 --program-suffix=-6.2.0 --with-arch=armv6 --with-fpu=vfp --with-float=hard --build=arm-linux-gnueabihf --host=arm-linux-gnueabihf --target=arm-linux-gnueabihf
			
			vncserver -kill :1
			
			
		\end{verbatim}
		
		Modify swap file as in link, then:
		
		\begin{verbatim}
		sudo /etc/init.d/dphys-swapfile stop
		sudo /etc/init.d/dphys-swapfile start

	cd ~
	cd build
	make -j 4
		\end{verbatim}
		
		\subsection{Install QT 5.7}
		Run:
		\begin{verbatim}
			sudo apt-get update
			sudo apt-get upgrade
			sudo apt-get install libfontconfig1-dev libdbus-1-dev libfreetype6-dev libudev-dev libicu-dev libsqlite3-dev libxslt1-dev libssl-dev libasound2-dev libavcodec-dev libavformat-dev libswscale-dev libgstreamer0.10-dev libgstreamer-plugins-base0.10-dev gstreamer-tools gstreamer0.10-plugins-good gstreamer0.10-plugins-bad libraspberrypi-dev libpulse-dev libx11-dev libglib2.0-dev libcups2-dev freetds-dev libsqlite0-dev libpq-dev libiodbc2-dev libmysqlclient-dev firebird-dev libpng12-dev libjpeg9-dev libgst-dev libxext-dev libxcb1 libxcb1-dev libx11-xcb1 libx11-xcb-dev libxcb-keysyms1 libxcb-keysyms1-dev libxcb-image0 libxcb-image0-dev libxcb-shm0 libxcb-shm0-dev libxcb-icccm4 libxcb-icccm4-dev libxcb-render-util0 libxcb-render-util0-dev libxcb-xfixes0-dev libxrender-dev libxcb-shape0-dev libxcb-randr0-dev libxcb-glx0-dev libxi-dev libdrm-dev libssl-dev libxcb-sync1 libxcb-sync-dev libxcb-xinerama0 libxcb-xinerama0-dev libjpeg62-turbo-dev
			
		\end{verbatim}
		
		\begin{verbatim}
			sudo apt-get update
			sudo apt-get ugrade
			sudo apt-get install qtcreator qt5-default qt5-qmake qtbase5-dev-tools
		\end{verbatim}
		
		
		
		

		
	

	\begin{versionhistory}
		\vhEntry{1.0}{11.19.16}{WL}{Created}
	\end{versionhistory}
		
		
\end{document}

